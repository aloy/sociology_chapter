\documentclass{article} % JASA requires 12 pt font for manuscripts
%\usepackage{JASA_manu}        % For JASA manuscript formatting

% for citations
\usepackage[authoryear]{natbib} % natbib required for JASA
\usepackage[colorlinks=true, citecolor=blue, linkcolor=blue]{hyperref}

%\definecolor{Blue}{rgb}{0,0,0.5}

% fonts
\usepackage{kpfonts}

% for figures
\usepackage{graphicx}
\graphicspath{{figures/}}

% help with editing and coauthoring
\usepackage{todonotes}

% title formatting
\usepackage[compact,small]{titlesec}

% page formatting
\usepackage[margin = 1in]{geometry}
\usepackage[parfill]{parskip}

% line spacing
\usepackage{setspace}
\doublespace

% For math typsetting
\usepackage{bm}
\usepackage{amstext}
\usepackage{amssymb}
\usepackage{amsmath}
\usepackage{amsfonts}
\usepackage{multirow}

% A few commands to make typing less tedious
\newcommand{\inv}{\ensuremath{^{-1}}}
\newcommand{\ginv}{\ensuremath{^{-}}}
\newcommand{\trans}{\ensuremath{^\prime}}
\newcommand{\var}{\ensuremath{\mathrm{Var}}}
\newcommand{\cov}{\ensuremath{\mathrm{Cov}}}


\title{Simulation-Based Diagnostics for Hierarchical Linear Models: A Graphical Approach}

\author{Adam Loy and Heike Hofmann}
%	Department of Statistics\\
%	Iowa State University\\
%	Ames, IA 50011-1210
%}

\begin{document}
\maketitle
%------------------------------------------------------------------------------------
\section{Introduction}
%------------------------------------------------------------------------------------

Hierarchical linear models (i.e., multilevel models, linear mixed effects models, random coefficient models) are versatile models that allow for dependence expected when data are organized in hierarchical structures---such data structures are especially common in social science research where studies often focus on responses from human subjects. The additional flexibility offered by these models to incorporate data from each level of the data hierarchy and allow for dependencies between individuals within the same group complicates model exploration and verification. More specifically, residual plots often display noticeable patterns, however, these patterns are often artifacts of the estimation procedure rather than indications of lack of fit. To mitigate the detection of such artifacts we suggest a graphical simulation-based approach to model checking that makes use of the protocols from visual inference \citep{Buja:2009hp}.




%------------------------------------------------------------------------------------
\section{Hierarchical linear models}\label{sec:hlms}
%------------------------------------------------------------------------------------

\subsection{Notation and formulation}

\subsection{Estimation}

\subsection{Residuals}

%------------------------------------------------------------------------------------
\section{Visual inference}\label{sec:vi}
%------------------------------------------------------------------------------------
In order to develop graphical model diagnostics as substitutes for statistical tests of model assumptions, we work within a rigorous inferential framework. \cite{Gelman:2004gg} formulates a visual analog of simulation-based model diagnostics in which a visualization of an aspect of the model is compared to data generated under the model. \cite{Buja:2009hp} extend this idea, proposing two protocols that formalize a rigorous inferential framework for testing visual discoveries. In this section we outline the lineup protocol.

Classical statistical inference consists of 
\begin{enumerate}
	\item formulating a null and alternative hypothesis,
	\item calculating a test statistic,
	\item comparing the test statistic to a reference (null) distribution,
	\item and calculating a $p$-value from on which we base our conclusion.
\end{enumerate}
Following \citeauthor{Buja:2009hp}, each of these components has a direct analog in visual inference for model diagnostics. The null and alternative hypotheses correspond to the model assumption, such as homogeneity of residual variance, and any violation of this model assumption, respectively. The test statistic corresponds to a plot that displays the model assumption. In visual inference, the reference distribution would consist of the set of all plots generated under the assumed model, called null plots. If the model assumptions are upheld, the plot of the observed data should indistinguishable from the null plots. A lineup of plots consists of displaying a number of null plots within which the true plot, that is, the test statistic, is embedded. Adhering to this lineup protocol allows for the definition of a $p$-value similar to that of permutation tests---if the true plot is distinguishable from a group of $m$ null plots, we would reject the null hypothesis with a $p$-value of $1 / (k + 1)$.

%------------------------------------------------------------------------------------
\section{Examples}
%------------------------------------------------------------------------------------
Applications of hierarchical models:
\begin{itemize}
\item Education -- students nested in classes nested in schools
\item Interviewer research -- respondents nested in interviewer (Hox 1994)
\end{itemize}

Situations to consider:
\begin{itemize}
\item Residual analysis
	\begin{itemize}
	\item Unbalanced sample sizes introduces structure
	\item Using raw rather than standardized residuals can introduce structure
	\item This could help avoid using theoretical cutoffs for statistics used in the detection of outliers and influential points. \cite{Longford:2001wy} discusses numeric simulation-based diagnostics for outlier and makes some good points about how the theoretical distributions of these statistics are: (1) hard to calculate theoretically in many situations; (2) are based on a unit being randomly selected, as opposed to selected after an inspection of the data.
	\end{itemize}

\item Comparison of random effects -- \cite{Morrell:2000ve} discuss lines in plots of random effects appear when there are groups with a large amount of pooling (shrinkage)

\item Exploratory modeling
	\begin{itemize}
	\item Considering a variable for inclusion in the model as a fixed effect
	\item The need for/utility of additional random effects
	
	\end{itemize}


\end{itemize}

%------------------------------------------------------------------------------------
\section{Discussion}
%------------------------------------------------------------------------------------


%------------------------------------------------------------------------------------
%------------------------------------------------------------------------------------

\bibliographystyle{apalike}
\bibliography{hlmviz_bib}

\end{document}